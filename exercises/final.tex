\documentclass[12pt]{article}
\usepackage{natbib,amssymb,amsmath,amsthm,epsfig,color,verbatim}
\newcommand{\sko}{\vspace{.2in}}
\setlength{\topmargin}{-.7in}
\setlength{\oddsidemargin}{0.3in}
\setlength{\textwidth}{6.15in}
\setlength{\textheight}{9.2in}
%\renewcommand{\baselinestretch}{1.2}

\begin{document}

\title{\bf  SDS 383D Final exam}
%\author{\bf SDS 383D}
\date{\it Spring 2018} %change for each day.
\maketitle{}
For all questions, you do not need to repeat calculations that you have carried out in homework exercises; instead you may refer back to the appropriate exercise. Any derivations that do not refer back to homework exercises must be described in full. 
\begin{enumerate}
\item Let
  $$\begin{aligned}
  \lambda \sim& \mbox{Gamma}(\alpha,\beta) \\
  X_i \sim& \mbox{Poisson}(\lambda)
\end{aligned}$$
  Derive the form of $P(X_{n+1}|X_{1:n})$
\item Now, let $X$ be sampled from a mixture of two Poisson distributions, so that
  $$\begin{aligned}
  \pi \sim& \mbox{Beta}(\tau,\tau)\\
  \lambda_k \sim& \mbox{Gamma}(\alpha,\beta),\qquad k \in\{0,1\}\\
  Z_i|\pi \sim& \mbox{Bernoulli}(\pi)\\
  X_i|Z_i=k \sim& \mbox{Poisson}(\lambda_k)
\end{aligned}$$
  Derive the form of $P(Z_i|Z_{-i},X)$
\item Let
  $$\begin{aligned}
  y \sim& \mbox{Normal}(X\beta, I)\\
  \beta \sim& \mbox{Normal}(0,(X^TX)^{-1})
\end{aligned}$$
  Showing all work (i.e.\ \textit{not} simply referring back to previous exercise questions), what is the posterior distribution over $\beta$ given $y$?
\item You want to use a Gaussian process
  $$f\sim \mbox{GP}(0,K)$$
  to model annual rainfall at various geographic locations. Since you know that rainfall is always positive, a Gaussian likelihood is not appropriate here. Write down an appropriate likelihood model for this problem (there are a number of valid options!), and describe either:
  \begin{enumerate}
  \item A Gibbs sampler for posterior inference of the latent function $f$, providing all necessary conditional likelihoods.
  \item Or, the steps required (i.e. an equation that must be solved to provide the mean, and an analytic form for the covariance) for a Laplace approximation to the posterior of $f$.
  \end{enumerate}
\end{enumerate}
\end{document}
