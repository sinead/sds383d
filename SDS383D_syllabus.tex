% This syllabus template was created by:
% Brian R. Hall
% Assistant Professor, Champlain College
% www.brianrhall.net

% Document settings
\documentclass[11pt]{article}
\usepackage[margin=1in]{geometry}
\usepackage[pdftex]{graphicx}
\usepackage{multirow}
\usepackage{multicol}
\usepackage{setspace}
\pagestyle{plain}

%\setlength\parindent{0pt}
\usepackage[parfill]{parskip}
\begin{document}

\newcommand{\Prob}{\mathbf{P}}

% Course information
\includegraphics[width=\textwidth]{sdslogo.png}


% Professor information
\begin{tabular}{ p{3in} l }
 \multicolumn{2}{l}{\LARGE SDS 383D Statistical Modeling II} \\\\
   \Large Syllabus (Spring 2018)\\\\\\\\
  \textbf{Professor} & Sinead Williamson\\
  \textbf{Office Hours} & Friday, 3:30-5:30\\
  \textbf{Location} & GDC 7.508\\
  \textbf{Email} & \texttt{Sinead.Williamson@mccombs.utexas.edu}\\
\end{tabular}


% Course details

\vspace*{1in}

\textbf {\large \\ Course Description}


This course is a PhD-level course on statistical modeling, focusing on a hierarchical Bayesian framework. The goal of the course is to develop the mathematical tools and intuition required to develop and deploy sophisticated Bayesian models. Major topics covered will be the multivariate normal distribution, hierarchical regression models, and latent variable models. 

The course will be run along the lines of Moore's method, with much of the in-class time spent with students explaining and exploring the concepts covered. While I will spend some time giving traditional lectures, much of the learning will be guided by you, the student. Outside of class, you will work on exercises which we will then discuss and explore in class, allowing everyone to come to a deep understanding of the material.

\vspace{5mm}

\textbf{Course structure}

Each week, I will give you a set of exercises and light notes. These will contain results to prove, problems to solve or suggest approaches to, and datasets to analyse. In class, you will present your solutions and we will discuss the material. This will be supplemented by more traditional lectures where necessary.

You should expect to spend a substantial amount of time -- typically 5-10 hours a week -- outside class working on the exercises. I encourage you to work on your own to obtain maximum benefit. 

\newpage

\textbf{Prerequisites}

The official pre-requisite for the course is SDS 383C. More generally, I assume familiarity with the following:
\begin{enumerate}
\item Linear algebra and multivariate calculus.
\item Basic programming skills in an appropriate language. The official programming language is R, but Python or Matlab would also be perfectly acceptable.
\item Basic probability and statistics.
  \begin{itemize}
    \item For probability, I expect the level of ``A First Course in Probability'' by Ross, or ``Introduction
      to Probability'' by Bertsekas and Tsitsiklis. You do not need measure theory or stochastic processes.
    \item For statistics, I expect the level of ``Statistical Inference'' by Casella and Berger. In particular, I expect you to be familiar with sampling distributions, confidence intervals, hypothesis testing, point estimation, and the basics of Bayesian inference.
  \end{itemize}
\item A basic understanding of linear regression, including at least some exposure to generalized linear models.
\end{enumerate}
  

\vspace{5mm}


\textbf{\large Course materials}

Exercises, datasets and any additional material will be posted on github. You are expected to create your own github repository where you will upload your work. 

There is not an officially required course book, but I recommend ``Bayesian Data Analysis'' by Gelman et al.


\textbf {\large Evaluation}

Your grade consists of three elements: 50\% peer evaluation, 30\% final project, and 20\% final exam.


\textbf{Peer evaluation}

Throughout the semester, you will upload to github your exercise solutions and any code generated. Exercise solutions should be typeset using Latex, and code should be documented. In class, you will present your work, either on the board or by connecting your laptop to the projector.

The first week of each month, you will randomly be assigned another student to peer review. To do this, you will look at their github and provide constructive criticism of their solutions and code. At the end of the semester, you will prepare a (confidential) report on each of the other students' contributions to the course. I will use these reports to inform the peer-evaluation grades.


\textbf{Project}

For the project part of the course, you may pick a topic of interest that show-cases the topics discussed in class, after discussing it with me in advance. This might be analysing a dataset associated with your own research, or developing and implementing a new model or inference technique and comparing it with existing methods. You may work in a group of three or fewer. 

You should prepare a report up to 8 pages in length using the NIPS style files and guidelines, available here: \texttt{https://nips.cc/Conferences/2017/PaperInformation/StyleFiles}. Reports are due on the final day of class.

We will have three in-class exams, on 2/21, 3/28 and 5/4. If you have to miss an exam due to a legitimate documented conflict, please let me know as soon as possible, and we will make arrangements for a make-up exam. 



\textbf{Exam}

There will be a final take-home exam at the end of the semester.



\vspace{5mm}

\textbf {\large Requests for Regrade}

Clerical requests will be corrected without hassle. Other regrading requests must be submitted in writing within one week (7 days) of the exam's return. Be aware that the entire exam will be subject to regrading, and grades may go up or down.



%\textbf{\Large Important Notifications}\\

\vspace{5mm}

\textbf{\large Students with Disabilities }

Students with disabilities may request appropriate academic accommodations from the Division of Diversity and Community Engagement, Services for Students with Disabilities, 512-471-6259, \texttt{http://www.utexas.edu/diversity/ddce/ssd/}.

\vspace{3mm}
\textbf{\large Religious Holy Days}

By UT Austin policy, you must notify me of your pending absence at least fourteen days prior to the date of observance of a religious holy day.  If you must miss a class, an examination, a work assignment, or a project in order to observe a religious holy day, you will be given an opportunity to complete the missed work within a reasonable time after the absence.

\vspace{3mm}
\textbf{\large Scholastic Honesty}

We expect students to behave with integrity.  Students found cheating on exams or homework will receive a score of zero for that exam or assignment, and may be subject to additional disciplinary action. For more information on the University of Texas scholastic dishonesty policy, see the 2006-2007 General Information Catalog, Appendix C.

%The McCombs School of Business has no tolerance for acts of scholastic dishonesty.  The responsibilities of both students and faculty with regard to scholastic dishonesty are described in detail in the BBA Program’s Statement on Scholastic Dishonesty at \texttt{http://www.mccombs.utexas.edu/BBA/} \texttt{Code-of-Ethics.aspx}.  By teaching this course, I have agreed to observe all faculty responsibilities described in that document. By enrolling in this class, you have agreed to observe all student responsibilities described in that document.  If the application of the Statement on Scholastic Dishonesty to this class or its assignments is unclear in any way, it is your responsibility to ask me for clarification.  Students who violate University rules on scholastic dishonesty are subject to disciplinary penalties, including the possibility of failure in the course and/or dismissal from the University.  Since dishonesty harms the individual, all students, the integrity of the University, and the value of our academic brand, policies on scholastic dishonesty will be strictly enforced.  You should refer to the Student Judicial Services website at \texttt{http://deanofstudents.utexas.edu/sjs/} to access the official University policies and procedures on scholastic dishonesty as well as further elaboration on what constitutes scholastic dishonesty. 

\vspace{3mm}
\textbf{\large Campus Safety}

Please note the following recommendations regarding emergency evacuation from the Office of Campus Safety and Security, 512-471-5767, \texttt{http://www.utexas.edu/safety}:
\begin{itemize}
\item Occupants of buildings on The University of Texas at Austin campus are required to evacuate buildings when a fire alarm is activated.  Alarm activation or announcement requires exiting and assembling outside.
\item Familiarize yourself with all exit doors of each classroom and building you may occupy.  Remember that the nearest exit door may not be the one you used when entering the building.
\item Students requiring assistance in evacuation should inform the instructor in writing during the first week of class.
\item In the event of an evacuation, follow the instruction of faculty or class instructors.
\item Do not re-enter a building unless given instructions by the following: Austin Fire Department, The University of Texas at Austin Police Department, or Fire Prevention Services office.
\item Behavior Concerns Advice Line (BCAL):  512-232-5050
\item Further information regarding emergency evacuation routes and emergency procedures can be found at: \texttt{http://www.utexas.edu/emergency}.
\end{itemize}


\end{document}



